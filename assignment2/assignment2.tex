\documentclass[12pt,a4paper]{report}
\usepackage[utf8]{inputenc}
\usepackage{amsmath}
\usepackage{amsfonts}
\usepackage{amssymb}
\usepackage{color}
\author{Malcolm Watt}
\title{Numerical Methods Assignment 2}
\usepackage{sectsty}
\begin{document}

\maketitle
\section*{Chaper 3 Exercises}
\subsection*{Question 3.3}
The linear least squares system is given as follows:
{\color{blue}
$$A x= 
\begin{bmatrix}
1 & e\\
2 & e^2\\
3 & e^3
\end{bmatrix}
\begin{bmatrix}
x_1 \\ x_2
\end{bmatrix}
\approx
\begin{bmatrix}
2 \\
3 \\
5
\end{bmatrix}
= b
$$
}
\subsection*{Question 3.18}
Consider the vector $$ a = \begin{bmatrix}2\\3\\4\end{bmatrix}$$.
\subsubsection{(a)}
We need to annihilate the third component of a, using an elementary elimination
matrix. The easiest is to subtract twict the first column from the last column.
Therefore the elementary row elimination matrix is given as
{\color{blue}
$$\begin{bmatrix}
1 & 0 & 0\\
0 & 1 & 0\\
-2 & 0 & 1\\
\end{bmatrix}$$
}
\subsubsection{(b)}
Now we need to do the same thing but with a Householder transformation.
We want to apply two Householder transformations, the first $H_1$ eliminating all
but the first element, and the second $H_2$ eliminating all but the second element.
Then it stand that the Householder matrix $H$ which eliminates only the third
element is simply the matrix multiplication of $H = H_1H_2$.

Let's find $H_1$
$$
v = a - \alpha (e_1) =
\begin{bmatrix}
2\\
3\\
4
\end{bmatrix}
- \alpha
\begin{bmatrix}
1\\
0\\
0
\end{bmatrix}
$$
where $\alpha = \pm \lvert \lvert a \rvert \rvert_2 = \pm \sqrt{29}$. Since $a_1$ is positive, we should use the negative $\alpha$ to avoid cancellation.


Following this we have
$$
v = a - \sqrt{29}
\begin{bmatrix}
1\\
0\\
0
\end{bmatrix}
=
\begin{bmatrix}
2 - \sqrt{29}\\
3\\
4
\end{bmatrix}
$$

and Householder matrix

$$H_1 = I - 2\frac{vv^T}{v^Tv} = 
\begin{bmatrix}
0.6885828  & 0.37139068& -0.62283441\\
0.37139068 & 0.55708601 & 0.74278135\\
-0.62283441 & 0.74278135& -0.24566881
\end{bmatrix}
$$

by a similar process we find 
$$
H_2 =
\begin{bmatrix}
  1.45796221 & 0.64642473&-1.19938586\\
 0.64642473 & 1.22908525 & 1.43036648\\
-1.19938586 & 1.43036648 &-0.68704746
\end{bmatrix}
$$

and the desired 

{\color{blue}
$$H = H_1 + H_2 = 
\begin{bmatrix}
 1.37716559  &0.74278135& -1.24566881\\
  0.74278135  &1.11417203 & 1.48556271\\
 -1.24566881  &1.48556271 &-0.49133762
\end{bmatrix}
$$
}
\subsubsection{(c)}
We pick the Given's Rotation matrix such that G is of the form
$$
\begin{bmatrix}
1 & 0 & 0\\
0 & c & s\\
0 & -s & c
\end{bmatrix}
$$

Since $\lvert a_3 \rvert > \lvert a_2 \rvert$ we use the cotangent formulation
to find $c$ and $s$. We have $\tau = c / s = \frac{a_3}{a_2} = 0.75$ and
therefore
$$s = \frac{1}{\sqrt{1+ \tau^2}} =\frac{4}{5}$$
and
$$c = \frac{3}{5}.$$

so the givens rotation matrix is

{\color{blue}
$$G = \begin{bmatrix}
1&0&0\\
0&\frac{3}{5}&\frac{4}{5}\\
0&\frac{-4}{5}&\frac{3}{5}
\end{bmatrix}
$$
}

{\color{red}
\subsection*{Question 3.24}
\subsection*{Question 3.25}
}
\end{document}