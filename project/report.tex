\documentclass[12pt]{article}
 \usepackage[margin=1in]{geometry} 
\usepackage{amsmath,amsthm,amssymb,amsfonts}
 \usepackage{color}
\newcommand{\N}{\mathbb{N}}
\newcommand{\Z}{\mathbb{Z}}
 \usepackage{amsmath}

\newenvironment{problem}[2][Problem]{\begin{trivlist}
\item[\hskip \labelsep {\bfseries #1}\hskip \labelsep {\bfseries #2.}]}{\end{trivlist}}
%If you want to title your bold things something different just make another thing exactly like this but replace "problem" with the name of the thing you want, like theorem or lemma or whatever
 
\begin{document}
 
%\renewcommand{\qedsymbol}{\filledbox}
%Good resources for looking up how to do stuff:
%Binary operators: http://www.access2science.com/latex/Binary.html
%General help: http://en.wikibooks.org/wiki/LaTeX/Mathematics
%Or just google stuff
 
\title{Numerical Methods Final Project}
\author{Malcolm Watt}
\maketitle
 
\begin{problem}{7.1}
Given the three data points (-1, 1),(0, 0), (1, 1), determine the interpolating polynomial
of degree two:
\begin{itemize}
    \item (a) Using the monomial basis
\item (b) Using the Lagrange basis
\item (c) Using the Newton basis

\end{itemize}
Show that the three representations give the
same polynomial.

\end{problem}
 
 \textit{(a) Monomial Basis}\\
 The resulting polynomial is of a maximum degree of 2.
 
 $$\phi_j(t)=t^{j-1}, \quad j=1,...,3$$
 Therefore the interpolating polynomial has the form:
 $$p_{n-1}(t) = x_1 + x_2 t + x_3 t^2.$$
 
The coefficients are determined by the $3x3$ linear system:
$$\left(\begin{smallmatrix}
1 & t_1 & t_1^2\\
1 & t_2 & t_2^2\\
1 & t_3 & t_3^2
\end{smallmatrix}\right)
\left(\begin{smallmatrix}
x_1\\x_2\\x_3
\end{smallmatrix}\right)
=\left(\begin{smallmatrix}
y_1\\y_2\\y_3
\end{smallmatrix}\right)$$

For our particular case:
$$\left(\begin{smallmatrix}
1 & -1 & 1\\
1 & 0 & 0\\
1 & 1 & 1
\end{smallmatrix}\right)
\left(\begin{smallmatrix}
x_1\\x_2\\x_3
\end{smallmatrix}\right)
=\left(\begin{smallmatrix}
1\\0\\1
\end{smallmatrix}\right)$$

Solving for the x vector we have 
$$\left(\begin{smallmatrix}
x_1\\x_2\\x_3
\end{smallmatrix}\right)=
\left(\begin{smallmatrix}
0\\0\\1
\end{smallmatrix}\right)$$
 
 Therefore our resulting polynomial is $$p_2(t)=t^2.$$
 
 \textit{(b) Lagrange Basis}
 $$p_2(t) = y_1\frac{(t-t_2)(t-t_3)}{(t_1-t_2)(t_1-t_3)}+y_2\frac{(t-t_1)(t-t_3)}{t_2-t_1)(t_2-t_3}+y_3\frac{(t-t_1)(t-t_2)}{(t_3-t_1)(t_3-t_2)}.$$
 
 For the data given this is
 $$p_2(t) = 1\frac{(t-0)(t-1)}{(-1-0)(-1-1)}+0\frac{(t+1)(t-1)}{(0+1)(0-1)}+1\frac{(t+1)(t-0)}{(1+1)(1-0)}.$$
 
 $$p_2(t) = \frac{t^2-t}{2}+\frac{t^2+t}{2} = t^2.$$
 
 \textit{(c) Newton Basis}
 Our polynomial is of the form
 $$p_2(t)=x_1+x_2(t-t_1)+x_3(t-t_1)(t-t_2).$$
 The basis functions are given by
 $$\phi_j(t) = \prod_{k=1}^{j-1}(t-t_k), \quad j= 1,...,n,$$
 
 Giving us the following triangular system:
 $$\left(\begin{smallmatrix}
1 & 0 & 0\\
1 & t_2-t_1 & 0\\
1 & t_3-t_1 & (t_3-t_1)(t_3-t_2)
\end{smallmatrix}\right)
\left(\begin{smallmatrix}
x_1\\x_2\\x_3
\end{smallmatrix}\right)
=\left(\begin{smallmatrix}
y_1\\y_2\\y_3
\end{smallmatrix}\right)$$

which for our particular case gives us
$$\left(\begin{smallmatrix}
1 & 0 & 0\\
1 & 1 & 0\\
1 & 2 & 2
\end{smallmatrix}\right)
\left(\begin{smallmatrix}
x_1\\x_2\\x_3
\end{smallmatrix}\right)
=\left(\begin{smallmatrix}
1\\0\\1
\end{smallmatrix}\right)$$

Which yields
$$\left(\begin{smallmatrix}
x_1\\x_2\\x_3
\end{smallmatrix}\right)=
\left(\begin{smallmatrix}
1\\-1\\1
\end{smallmatrix}\right)$$

Therefore our polynomial is
$$p_2(t) = 1-1(t+1)+1(t+1)(t)=1-1+t-t+t^2 = t^2.$$

\textbf{For every method, our resulting polynomial is} $$p_2(t)=t^2.$$

\begin{problem}{7.2}
Express the following polynomial in the correct form for evaluation by Horner’s method: $p_3(t) = 5t^3 - 3t^2 + 7t - 2$.

Horner's method is a pretty simple rearangement of the terms given as follows:
$$p_3(t)=x_1+t(x_2+t(x_3+x_4t))$$
Which for our particular case is:
$$p_3(t) = -2+t(7+t(-3+5t)).$$
\end{problem}

\begin{problem}{7.5}
 In general, is it possible to interpolate n data points by a piece-wise quadratic polynomial, with knots at the given data points, such that the interpolant is
 \begin{itemize}
     \item (a) Once continuously differentiable?
     \item (b) Twice continuously differentiable?
     \end{itemize}

In each case, if the answer is \textit{yes}, explain
why, and if the answer is 
\textit{no}, give the maximum
value for n for which it is possible.

\textit{(a) Once continuously differentiable}

Yes it is possible, for example, we can use the Hermite Cubic interpolant, which adds the derivatives at the knot points as paramters. This addition means we have 4(n-1) parameters to be determined as opposed to n.

\textit{(b) Twice continuously differentiable}
 Yes it is possible using \textit{cubic spline} interpolation. It adds $3n-4$ constraints for the first derivative and $n-2$ additional constraints for the second derivative.
\end{problem}

\begin{problem}{7.10}
 (a) Verify directly that the first six Legendre polynomials given in Section 7.2.4 are indeed mutually orthogonal. \\
 (b) Verify directly that they satisfy the three-term recurrence given in Section 7.2.4.\\
 (c) Express each of the first six monomials, $1, t, ..., t^5$, as a linear combination of the first six Legendre polynomials, $p_0, ..., p_5$.
 
 \textit{(a) Verify Legendre Polynomials}
 $1, t, (3t^2 - 1)/2, (5t^3 - 3t)/2, (35t^4 - 30t^2 + 3)/8, (63t^5 - 70t^3 +15t)/8$
 The set of polynomials is orthonormal if
 \[(p_i, p_j) = \begin{cases} 
      1 & i=j \\
      0 & i \neq j
   \end{cases}
\]

where $(p,q) = \int_a^b p(t)q(t)w(t)dt.$

For the Legendre polynomials, $w(t) \equiv 1$ and the interval is defined as $[a,b] = [-1,1].$

We will evaluate each integral in a table for readability.

\scriptsize
\begin{tabular}{|c|c|c|c|c|c| c|}
\hline
i&j=1&j=2&j=3&j=4&j=5&j=6\\
\hline
1&$\int_{-1}^{1}(1)(1)dt$&...&...&...&...&...\\
2&$\int_{-1}^{1}(1)(t)dt$&$\int_{-1}^{1}(t)(t)dt$&...&...&...&...\\
3&$\int_{-1}^{1}(1)(3t^2-1)/2dt$&$\int_{-1}^{1}(t)(3t^2-1)/2dt$&$\int_{-1}^{1}(3t^2-1)^2/4dt$&...&...&...\\
4&$\int_{-1}^{1}(1)(5t^3-3t)/2dt$&$\int_{-1}^{1}(t)(5t^3-3t)/2dt$&$\int_{-1}^{1}(3t^2-1)(5t^3-3t)/2dt$&$\int_{-1}^{1}(5t^3-3t)^2/4dt$\\
5&$\int_{-1}^{1}(1)(35t^4-30t^2+3)/8dt$&$\int_{-1}^{1}(t)(35t^4-30t^2+3)/8dt$&$\int_{-1}^{1}(3t^2-1)(35t^4-30t^2+3)/8dt$&$\int_{-1}^{1}(3t^2-1)(35t^4-30t^2+3)/16dt$\\
6&$\int_{-1}^{1}(1)(63t^5-70t^3+15t)/8dt$&$\int_{-1}^{1}(t)(63t^5-70t^3+15t)/8dt$&$\int_{-1}^{1}(3t^2-1)(63t^5-70t^3+15t)/8dt$\\
\hline
\end{tabular}

\begin{tabular}{|c|c|c|c|c|c| c|}
\hline
i&j=1&j=2&j=3&j=4&j=5&j=6\\
\hline
1&$t\vert_{-1}^{1}$&...&...&...&...&...\\
2&$\frac{t^2}{2}\vert_{-1}^{1}$&$\frac{t^3}{3}\vert_{-1}^{1}$&...&...&...&...\\
3&$t^3-1\vert_{-1}^{1}$&$\int_{-1}^{1}(t)(3t^2-1)/2dt$&$\int_{-1}^{1}(3t^2-1)^2/4dt$&...&...&...\\
4&$\frac{5}{4}t^4-\frac{3}{2}t^2\vert_{-1}^{1}$&$\int_{-1}^{1}(t)(5t^3-3t)/2dt$&$\int_{-1}^{1}(3t^2-1)(5t^3-3t)/2dt$&$\int_{-1}^{1}(5t^3-3t)^2/4dt$\\
5&$\frac{7t^5-10t^3+3t}{8}\vert_{-1}^{1}$&$\int_{-1}^{1}(t)(35t^4-30t^2+3)/8dt$&$\int_{-1}^{1}(3t^2-1)(35t^4-30t^2+3)/8dt$&$\int_{-1}^{1}(3t^2-1)(35t^4-30t^2+3)/16dt$\\
6&$\frac{\frac{63}{6}t^6-\frac{70}{4}t^4+\frac{15}{2}t^2}{8}\vert_{-1}^{1}$&$\int_{-1}^{1}(t)(63t^5-70t^3+15t)/8dt$&$\int_{-1}^{1}(3t^2-1)(63t^5-70t^3+15t)/8dt$\\
\hline
\end{tabular}

\begin{tabular}{|c|c|c|c|c|c| c|}
\hline
i&j=1&j=2&j=3&j=4&j=5&j=6\\
\hline
1&2&...&...&...&...&...\\
2&0&$\frac{2}{3}$&...&...&...&...\\
3&0&0&$\int_{-1}^{1}(3t^2-1)^2/4dt$&...&...&...\\
4&0&0&$\int_{-1}^{1}(3t^2-1)(5t^3-3t)/2dt$&$\int_{-1}^{1}(5t^3-3t)^2/4dt$\\
5&0&0&$\int_{-1}^{1}(3t^2-1)(35t^4-30t^2+3)/8dt$&$\int_{-1}^{1}(3t^2-1)(35t^4-30t^2+3)/16dt$\\
6&0&0&$\int_{-1}^{1}(3t^2-1)(63t^5-70t^3+15t)/8dt$\\
\hline
\end{tabular}

\normalsize
{
\color{red}
The basic idea is all of the diagonal integrals will be non-zero, but all of the other ones will be zero, so we have orthogonality.}

\textit{(b)} Starting with the first polynomial $p_0(t) = 1$ and with the recurrence formula given by
$$p_{k+1}(t)=\frac{(2k+1)tp_k(t)-kp_{k-1}(t)}{k+1}$$

$$p_0(t) = 1$$

$$p_1(t)=tp_0(t)=t$$

$$p_2(t) = \frac{3tp_1(t)-p_0(t)}{2}=\frac{3t^2-1}{2}$$
$$p_3(t) = \frac{5tp_2(t)-2p_1(t)}{3}=\frac{\frac{15t^3 - 5t}{2}-2t}{3}=\frac{5t^3-3t}{2}$$

$$p_4(t)=\frac{7t\frac{5t^3-3t}{2}-3\frac{3t^2-1}{2}}{4}=\frac{35t^4-30t^2-3}{8}$$

$$p_5(t)
=\frac{9t\frac{35t^4-30t^2-3}{8}-4\frac{5t^3-3t}{2}}{5}
=\frac{63t^5-70t^3+15t}{8}$$


\textit{(c) Expression of polynomials.}

$$t = p_1(t)$$
$$t^2 = \frac{2}{3}p_2(t)+\frac{2}{6} p_0(t)$$
$$t^3 = \frac{2}{5}p_3(t)+\frac{3}{5}p_1(t)$$
$$t^4 = \frac{8}{35}p_4(t)+\frac{20}{35}p_2(t) +\frac{17}{24}p_0(t)$$
$$t^5 = \frac{8}{63}p_5(t)
+\frac{28}{63}p_3(t)
+\frac{19}{21}p_1(t)$$
\end{problem}

\begin{problem}{7.11}
Verify the properties of B-splines enumerated
in Section 7.3.4.

The following is just a review of the principles:
We assume an infinite set of knots:
$$...<t_{-2}<t_{-1}<t_0<t_1<t_2<...$$

We make use of the linear functions:
$$v_i^k(t)=\frac{t-t_i}{t_{i+1}-t_i}.$$

We define B-splines of degree 0 by
\[B_i^0(t) = \begin{cases} 
      1 & if \quad t_i \leq t < t_{i+1} \\
      0 & otherwise
   \end{cases}
\]
and then for $k>0$ we define B-splines of degree k by

$$B_i^k(t) = v_i^k(t)B_i^{k-1}(t)+(1-v_{i+1}^k(t))B_{i+1}^{k-1}(t).$$



\textbf{Property 1} For $t<t_i$ or $t > t_{i+k+1}, \quad B_i^k(t)=0$

\begin{proof}

$$B_i^k(t) = v_i^k(t)B_i^{k-1}(t)+(1-v_{i+1}^k(t))B_{i+1}^{k-1}(t).$$

$B_i^k(t)$ is formed by the addition of scaled zero order B-splines $$B_j^0(t)$$ for $j \in [i, i + k + 1]$.

Given $t^* \not\in [i, i+k+1]$, for all $B_j^0(t)$ for $j \in [i, i + k + 1]$ we have $B_j^0(t^*)=0$. Therefore all the base terms of our recursion are zero and $B_i^k(t)$ is the addition of scaled zeros, therefore $B_i^k(t)=0$.
\end{proof}

{\color{blue}
\textbf{More verbose, but not really meaningful.. I spent too long on this to take it out :P.}
For $t<t_i$ or $t > t_{i+k+1}, \quad B_i^k(t)=0$

We can consider this domain of t variables as a special variable $t^*$ for notation purposes. The first thing to note is that the $B_i^0(t^*)$ is necessarily 0, because there is no t in the range of $t^*$ where we fall in the interval $t_i \leq t < t_{i+1}$.
Our recurrence relationship is defined as

$$B_i^k(t) = v_i^k(t)B_i^{k-1}(t)+(1-v_{i+1}^k(t))B_{i+1}^{k-1}(t).$$

Particularly, we will look at the k = 1 case to start, and argue why it extends for the rest of the problem.

The first term in the recursive definition is
$$v_i^k(t)B_i^{k-1}(t)$$

Which of course for the k=1 case, has the degree 0 spline term in it which has a value of zero, so this term goes to zero.

The second term requires a little more thought. Particularly the piece highlighted in red.

$$(1-v_{i+1}^k(t))\color{red}B_{i+1}^{k-1}(t)$$

This shifted spline term means we need to evaluate our zero degree spline over the shifted interval. However, note that in the recursive definition of our spline, we can shift a maximum of k times over, since at that point we hit the zero degree spline. Effectively, we can think of it as if our k degree polynomial has a non-zero zero order spline over the interval $[i, i+k)$.

That being said, if we are never in this range (which is the case for $t^*$), then we will never have a non-zero term in our 'travelling term', and therefore our recursion will always have zero valued terms, and will therefore be zero.

}
\textbf{Property 2} For $t_i < t < t_{i+k+1}, B_i^k(t)>0.$

This follows a very similar reasoning to the previous explanation.

\begin{proof}

$$B_i^k(t) = v_i^k(t)B_i^{k-1}(t)+(1-v_{i+1}^k(t))B_{i+1}^{k-1}(t).$$

$B_i^k(t)$ is formed by the addition of scaled zero order B-splines $$B_j^0(t)$$ for $j \in [i, i + k + 1]$.

Given $t^* \in [i, i+k+1]$, for \textbf{at least one} $B_j^0(t)$ for $j \in [i, i + k + 1]$ we have $B_j^0(t^*)=1$. The scalling factor will necessarily be a positive non-zero real number {\color{red} I should show this} therefore the $B_i^k(t)$ must be greater than 0.
\end{proof}

\textbf{Property 3} For all t, $\sum_{i=-\inf}^{\inf}B_i^k(t)=1$.

\begin{proof}
Ugh this is long and has to do with the fact that the scaling factor of $v_i^k(t)$ will accumulate to 1 over the range of i's where the term in the sum is non-negative.

Not much of a proof :(
\end{proof}

\end{problem}

\begin{problem}{8.2}
 (a) Using the composite midpoint quadrature rule, compute the approximate value for the integral $\int_0^1x^3dx$, using a mesh size (panel width) of h = 0.5 and also using a mesh size of h = 1.\\
(b) Based on the two approximate values computed in part a, use Richardson extrapolation to compute a more accurate approximation to the integral.\\
(c) Would you expect the extrapolated result computed in part b to be exact in this case? \textit{Why}?

\textit{(a)}
Composite midpoint quadrature rule, $\int_0^1x^3dx$.
Mesh size of 0.5:



\end{problem}

\begin{problem}{8.2,    8.5,    8.6,    8.9,    8.10}
\end{problem}

\end{document}